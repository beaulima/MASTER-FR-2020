\chapter*{Summary}

Due to the coherent nature of the Synthetic Aperture Radar (\textbf{SAR}) signal, polarimetric \textbf{SAR} (\textbf{POLSAR}) images are affected by speckle noise. The effect of speckle can be so severe as to render the \textbf{POLSAR} data unusable. This is especially true for single-look data that suffer from very intense speckle. Noise filtering is necessary to improve the estimation of polarimetric parameters that can be computed from this type of data. This is an important step in the processing and analysis of \textbf{POLSAR} images. 

Recently, a new approach has emerged in data processing aimed at solving a multitude of problems including filtering, image restoration, speech recognition, classification or image segmentation.  This approach is deep learning and convolutional neural networks (\textbf{CONVNET}). Recent works show that \textbf{CONVNET} are a promising alternative for filtering \textbf{SAR} images.  Indeed, by their ability to learn an optimal filtering model only from the data, they tend to outperform classical approaches to filtering on \textbf{SAR} images.

The objective of this study is to analyze and evaluate the effectiveness of \textbf{CONVNET} filtering on simulated \textbf{POLSAR} data and on RADARSAT-2, ALOS/PalSAR and GaoFen-3 satellite images acquired over the San Francisco urban area (California).   Models inspired by the architecture of a \textbf{CONVNET} used in particular in super-resolution have been adapted for the filtering of the polarimetric coherency matrix. The effect of different structural parameters of the \textbf{CONVNET} architecture on filtering were analyzed, among which are the depth of the neural network (the number of stacked layers), the width of the  neural network (the number of filters per convoluted layer) and the size of the filters of the first convolution layer.  

The models were learned by back-propagating the error gradient using 3 datasets that simulate single-look polarimetry of the scatterers according to Cloude-Pottier classes. The first dataset contains only homogeneous areas.  The last two datasets consist of patchwork simulations where local intensity is simulated by texture images and point target are added to the patchwork in the case of the last dataset. The performance of the different filters by \textbf{CONVNET} was measured by indicators including relative error on the estimation of polarimetric signatures and decomposition parameters as well as distortion measurements on the recovery of important details and on the conservation of point targets. 

The results show that \textbf{CONVNET} filtering of polarimetric data is either equivalent or significantly superior to conventional polarimetric filters.  The results of the deepest models obtain the best performance for all indicators over the simulated homogeneous dataset.  In the case of patchwork dataset, the results for detail restoration are clearly favourable to the deepest \textbf{CONVNET} filtering.

The application of \textbf{CONVNET} filtering on RADARSAT-2, ALOS/PalSAR and GaoFen-3 satellite images shows results comparable or superior to conventional filters.  The best results were obtained by the 5 hidden layers model (not counting the input and output layers), with 8 filters $3 \times 3$ per convolutional layer, except for the input layer where the filter size was  $9 \times 9$.  On the other hand, the training data must be well adjusted to the statistical range of the real polarimetric images to obtain good results.  This is especially true when modeling point targets that appear to be more difficult to restore.
\newline

\textbf{Key words:} Deep Learning, convolution neural network, polarimetry, synthetic aperture radar, filtering, speckle, polarimetric estimation, San Francisco.
