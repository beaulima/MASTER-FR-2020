\chapter*{Sommaire}

En raison de la nature cohérente du signal RADAR à synthèse d’ouverture (\acrsarns), les images \acrsar polarimétriques (\acrpolsarns) sont affectées par le bruit de chatoiement. L’effet du chatoiement peut être sévère au point de rendre inutilisable la donnée \acrpolsarns. Ceci est particulièrement vrai pour les données à une vue qui souffrent d’un chatoiement très intense. Un filtrage du bruit est nécessaire pour améliorer l’estimation des paramètres polarimétriques pouvant être calculés à partir de ce type de données. Cette opération constitue une étape importante dans le traitement et l’analyse des images \acrpolsarns. 

Récemment une nouvelle approche est apparue en traitement de données visant la solution d’une multitude de problèmes dont le filtrage, la restauration d’images, la reconnaissance de la parole, la classification ou la segmentation d’images.  Cette approche est l’apprentissage profond et les réseaux de neurones à convolution (\acrconvnetns). Des travaux récents montrent que les \acrconvnet sont une alternative prometteuse pour le filtrages des images \acrsarns.  En effet par leur capacité d’apprendre un modèle optimal de filtrage, ils tendent à surpasser les approches classiques du filtrage sur les images \acrsarns.

L'objectif de cette présente étude est d'analyser et d'évaluer l'efficacité du filtrage par \acrconvnet sur des données \acrpolsar simulées et sur des images satellitaires \acrpolsar RADARSAT-2, ALOS/PalSAR et GaoFen-3 acquises sur la région urbaine de San Francisco (Californie).   Des modèles inspirés de l'architecture d'un \acrconvnet utilisé notamment en Super-résolution ont été adaptés pour le filtrage de la matrice de cohérence polarimétrique. L'effet de différents paramètres structuraux de l'architecture des \acrconvnet sur le filtrage ont été analysés, parmi ceux-ci on retrouve entre autres la profondeur du réseau (le nombre de couches empilées), la largeur du réseau (le nombre de filtres par couches convolutives) et la taille des filtres de la première couche convolutive.  

L'apprentissage des modèles a été effectué par la rétropropagation du gradient de l'erreur en utilisant 3 ensembles de données qui simulent la polarimétrie une vue des diffuseurs selon les classes de Cloude-Pottier. Le premier ensemble ne comporte que des zones homogènes.  Les deux derniers ensembles sont composés de simulations en patchwork dont l'intensité locale est simulée par des images de texture et de cibles ponctuelles ajoutées au patchwork dans le cas du dernier ensemble. Les performances des différents filtres par \acrconvnet ont été mesurées par des indicateurs comprenant l'erreur relative sur l'estimation de signatures polarimétriques et des paramètres de décomposition ainsi que des mesures de distorsion sur la récupération des détails importants et sur la conservation des cibles ponctuelles. 

Les résultats montrent que le filtrage par \acrconvnet des données polarimétriques est soit équivalent ou nettement supérieur aux filtres conventionnellement utilisées en polarimétrie.  Les résultats des modèles les plus profonds obtiennent les meilleures performances pour tous les indicateurs sur l'ensemble des données homogènes simulées.  Dans le cas des données en patchwork, les résultats pour la restauration des détails sont nettement favorables au filtrage par \acrconvnet les plus profonds.

L'application du filtrage par \acrconvnet sur les images satellitaires RADARSAT-2, ALOS/PalSAR ainsi GaoFen-3 montre des résultats comparables ou supérieurs aux filtres conventionnels.  Les meilleurs résultats ont été obtenus par le modèle à 5 couches cachées (si on ne compte pas la couche d'entrée et de sortie), avec 8 filtres $3 \times 3$ par couche convolutive, sauf pour la couche d'entrée où la taille des filtres étaient de $9 \times 9$.  Par contre, les données d'apprentissage doivent être bien ajustées à l'étendue des statistiques des images polarimétriques réelles pour obtenir de bon résultats.  Ceci est surtout vrai au niveau de la modélisation des cibles ponctuelles dont la restauration semblent plus difficiles.
\newline

\textbf{Mots clés:} Apprentissage automatique, réseau de neurones à convolution, polarimétrie, RADAR à synthèse d’ouvertures, filtrage, chatoiement, estimation polarimétrique, San Francisco.


